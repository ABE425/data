\documentclass[11pt,letterpaper]{article}
\usepackage[margin=1.0in]{geometry}
\usepackage[utf8]{inputenc}
\usepackage{cite}
\usepackage{amsmath}
\usepackage{amsfonts}
\usepackage{amssymb}
\usepackage{makeidx}
\usepackage{graphicx}
\usepackage{hyperref}
\setlength\parindent{0pt}

\author{NAME:}
\title{HW: Operational Amplifiers.}

\begin{document}

\maketitle

You have a sensor that has a maximum output of 25 Volt, and you want to feed this signal into a data acquisition (DAQ) board with a maximum input in the range [0, +10] Volt. Design a circuit that matches the sensor to your DAQ board with the following design criteria:\\

a)	You need to attenuate the input signal by a factor of 2.5.\\
b)	The sensor cannot provide any current; therefore you need to make sure not to cause a loading error.\\ 
c)	You can only use a single OpAmp and two resistors (and everything else you need to make this work such as power supplies, wires etc.).\\ 

You consider the following three options: For each option, I) explain why each option does or does not work, and II) give the value of two resistors if the option works (choose one, calculate the other). Also, insert photos of HAND-DRAWN images here that show your attempts; you need figures for ALL THREE OPTIONS.

\begin{enumerate}
	\item Use a non-inverting OpAmp circuit with two feedback resistors. 
	\item Use an inverting OpAmp circuit with two feedback resistors.
	\item Design your own circuit (again use only one OpAmp and two resistors).
\end{enumerate}


\end{document}